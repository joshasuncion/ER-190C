\documentclass[12pt]{article}

\usepackage{graphics}
\usepackage{amsfonts}
\usepackage{empheq}
\usepackage{amsmath}
\usepackage{amssymb}
\usepackage{float}
\usepackage{steinmetz}
\usepackage{mathrsfs}
\usepackage{lipsum}
% \usepackage{undertilde}
\usepackage{ifthen}
\usepackage{hyperref}
\usepackage{framed}
\usepackage{makeidx}

\numberwithin{figure}{section}
\numberwithin{table}{section}
\numberwithin{equation}{section}

\renewcommand{\textfraction}{0}
\renewcommand{\topfraction}{1}
\renewcommand{\familydefault}{\sfdefault}

\newcommand{\lbf}{\textbf}
%\newcommand{\status}{nosolutions}

% \theoremstyle{example}
\newtheorem{example}{Example}[section]
\newcommand{\exmpl}[1]{\begin{shaded}\begin{example} #1 \end{example}\end{shaded}}
\newcommand{\frmd}[1]{\begin{framed} #1 \end{framed}}
\newcommand{\emq}[1]{\begin{empheq}[box=\fbox]{align*} #1 \end{empheq}}

\input{header.tex}
\usepackage{subfigure}
\usepackage{verbatim}
% \usepackage{hyperref}
\pdfpagewidth 8.5in
\pdfpageheight 11.0in

\usepackage{rotating}
\usepackage{pdflscape}

%\addtolength{\oddsidemargin}{0.5in}
%\addtolength{\evensidemargin}{0.5in}
%\addtolength{\textwidth}{-1in}
%\addtolength{\textheight}{-1.25in}
%\addtolength{\footskip}{0.15in}



\newcounter{lecture}
\usepackage{fancyhdr}
\pagestyle{fancy}
\lhead{}
\chead{}
\rhead{}
\lfoot{ER190C}
\rfoot{October 8, 2018}
\cfoot{\thepage}
\renewcommand{\headrulewidth}{0pt}
\renewcommand{\footrulewidth}{0pt}

\makeindex
\begin{document}
\definecolor{shadecolor}{gray}{0.85}

\begin{center}
  \textbf{ER90C, Data, Environment and Society}

  Fall 2018

  Lab \# 7

  Distributed: October 8, 2018

  Due: in class.

\end{center}


\begin{enumerate}
\item  Suppose you want to fit the model $y_i = \beta_0+\beta_1 x_i + \epsilon_i$ to the data below.  Draw your best guess for what the model would be: (a) using a solid line to represent the model you think a mean squared error loss function would give and (b) a dashed line for your guess at what you'd get with a a mean absolute error loss function.  Explain why you drew the lines as you did.

\includegraphics[height=0.25\textheight]{random_data}


\ifx\status\undefined
\begin{shaded}
The MSE line has a larger slope because it is more influenced by the extreme value at $(x,y) = (1.5,6)$.  
\end{shaded}
\else
\vspace*{4cm} 
\fi

\item Consider the following data set.  It shows how much people spent to send a package via the postal service.  The data also include the time zone they sent the package to and its weight.  Write down a model that could be used to predict shipping cost from the remaining data.  You may need to draw addtional columns in the dataframe to `code' categorical variables as we did in class.   


\includegraphics[height=0.35\textheight]{data_frame}

\ifx\status\undefined
\begin{shaded}
\includegraphics[height=0.35\textheight]{data_frame_solved}

$\text{cost} = \beta_0 + \beta_1 \times \text{weight} + \beta_2 x_{western} + \beta_3 x_{mountain} + \beta_4 x_{central} + \beta_5 x_{eastern}$

\end{shaded}
\else
\vspace*{4cm} 
\fi
\end{enumerate}

\end{document} 